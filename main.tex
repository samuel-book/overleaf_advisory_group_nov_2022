%%%%%%%%%%%%%%%%%%%%%%%%%%%%%%%%%%%%%%%%%%%%%%%%%%%%%%%%%%%%%%%

% Set up document

\documentclass{beamer}
\usecolortheme{whale}
\setbeamersize{text margin left=5mm,text margin right=5mm}

% Used to create a section slide between section
\AtBeginSection[]{
  \begin{frame}
  \vfill
  \centering
  \begin{beamercolorbox}[sep=8pt,center,shadow=true,rounded=true]{title}
    \usebeamerfont{title}\insertsectionhead\par%
  \end{beamercolorbox}
  \vfill
  \end{frame}
}

% Used to create a section slide between section
\AtBeginSection[]{
  \begin{frame}[noframenumbering, plain]
  \vfill
  \centering
  \begin{beamercolorbox}[sep=8pt,center,shadow=true,rounded=true]{title}
    \usebeamerfont{title}\insertsectionhead\par%
  \end{beamercolorbox}
  \vfill
  \end{frame}
}

% Used to create a subsection slide between subsections
% \AtBeginSubsubsection{\frame{\subsubsectionpage}}
\AtBeginSubsection[]{
  \begin{frame}[noframenumbering, plain]
  \vfill
  \centering
  \begin{beamercolorbox}[sep=8pt,center,shadow=true,rounded=true]{title}
    \usebeamerfont{title}\insertsectionhead:\par\phantom{space please}\par\insertsubsectionhead\par%
  \end{beamercolorbox}
  \vfill
  \end{frame}
}

% Remove default navigation symbols and add just  page number
\setbeamertemplate{navigation symbols}{} % Clear default navigation
\addtobeamertemplate{navigation symbols}{}{%
    \usebeamerfont{footline}%
    \usebeamercolor[fg]{footline}%
    \hspace{1em}%
    \insertframenumber/\inserttotalframenumber
}

%%%%%%%%%%%%%%%%%%%%%%%%%%%%%%%%%%%%%%%%%%%%%%%%%%%%%%%%%%%%%%%

% Title page

\title{Stroke Audit Machine Learning (SAMueL) \\ Advisory Group November 2022}
\subtitle{Investigating variation in thrombolysis use with clinical pathway simulation and explainable AI}


%\institute{Overleaf}
\date{November 2022}


\begin{document}

%\frame{\titlepage}

\begin{frame}
\titlepage

\end{frame}


%%%%%%%%%%%%%%%%%%%%%%%%%%%%%%%%%%%%%%%%%%%%%%%%%%%%%%%%%%%%%%%

\begin{frame}
\frametitle{Outline}
\tableofcontents
\end{frame}

%%%%%%%%%%%%%%%%%%%%%%%%%%%%%%%%%%%%%%%%%%%%%%%%%%%%%%%%%%%%%%%
\section{Modelled stroke pathway}

%%%%%%%%%%%%%%%%%%%%%%%%%%%%%%%%%%%%%%%%%%%%%%%%%%%%%%%%%%%%%%%

\begin{frame}
\frametitle{Breaking down the emergency stroke pathway into key steps}
\begin{center}
\includegraphics[width=1.0\textwidth]{./images/pathway}
\end{center}
We can model key changes to pathway:
\begin{small}
\begin{itemize}
    \item What if the pathway were faster?
    \item What if hospital determined the stroke onset time in more patients?
    \item What if clinical decision-making was like that of \emph{benchmark} hospitals? (Predict what treatment a patient would receive at other hospitals).
\end{itemize}
\end{small}
\footnotesize{We model these changes with a hospital's own patient population, to allow for inter-hospital variation in patient population characteristics.}
\end{frame}

%%%%%%%%%%%%%%%%%%%%%%%%%%%%%%%%%%%%%%%%%%%%%%%%%%%%%%%%%%%%%%%
\section{SAMueL-1 summary}


%%%%%%%%%%%%%%%%%%%%%%%%%%%%%%%%%%%%%%%%%%%%%%%%%%%%%%%%%%%%%%%

\begin{frame}
\frametitle{SAMueL-1 Summary: What is the problem?}
\begin{center}
\includegraphics[width=1.0\textwidth]{./images/sam_summary_pt_1}
\end{center}
\end{frame}

\begin{frame}
\frametitle{SAMueL-1 Summary: What did we test?}

We used clinical pathway simulation and machine learning to analyse a series of \emph{what if?} questions:

\begin{itemize}
    \setlength\itemsep{3mm}
    \item What if arrival-to-treatment time was 30 minutes?
    \item what if all hospitals determined stroke onset time as frequently as an \emph{upper quartile} hospital (a hospital ranked 25 out of 100, for determining stroke onset time).
    \item What if decisions to thrombolyse were made according to a majority vote of 30 benchmark hospitals?
\end{itemize}

For each hospital we use their own patients to ask these questions, to allow for differences in local patient populations.
\end{frame}

\begin{frame}
\frametitle{SAMueL-1 Summary: What did we find?}
\begin{center}
\includegraphics[width=1.0\textwidth]{./images/sam_summary_pt_3}
\end{center}
\end{frame}

%%%%%%%%%%%%%%%%%%%%%%%%%%%%%%%%%%%%%%%%%%%%%%%%%%%%%%%%%%%%%%%

\begin{frame}
\frametitle{Applying our models at hospital level}

\begin{center}
\includegraphics[width=0.95\textwidth]{./images/hosp_scenario_1}
\end{center}

\end{frame}

%%%%%%%%%%%%%%%%%%%%%%%%%%%%%%%%%%%%%%%%%%%%%%%%%%%%%%%%%%%%%%%
\section{Machine learning - learning and comparison decisions to thrombolyse between hopsitals}


%%%%%%%%%%%%%%%%%%%%%%%%%%%%%%%%%%%%%%%%%%%%%%%%%%%%%%%%%%%%%%%

\begin{frame}
\frametitle{Machine learning overview}
\begin{center}
\includegraphics[width=0.90\textwidth]{./images/ml_model_high_level}
\end{center}


Machine learning (and nearly all \emph{artificial intelligence}) is based on the simple principle of recognising similarity to what has been seen before.
\vspace{3mm}

We accessed 240,000 emergency stroke admissions in England and Wales over three years.
\end{frame}

%%%%%%%%%%%%%%%%%%%%%%%%%%%%%%%%%%%%%%%%%%%%%%%%%%%%%%%%%%%%%%%

\begin{frame}{Model accuracy, and simplification}

Our machine learning models use XGBoost classification, and are based on all patients who arrive within 4 hours of known stroke onset.

\vspace{5mm}

\begin{columns}[T] % [T] Top aligns columns

    \begin{column}{0.5\textwidth}
    
        The full model has 61 patient features:
        
        \begin{footnotesize}
        \begin{itemize}
            \item Overall accuracy = 85.2\%
            \item Best combined sensitivity and specificity = 84.3\%
            \item ROC AUC = 0.921
        \end{itemize}
        \end{footnotesize}
        
        \vspace{3mm}
        
        A simplified model with 8 features
        
        \begin{footnotesize}
        \begin{itemize}
            \item Overall accuracy = 84.8\%
            \item Best combined sensitivity and specificity = 83.8\%
            \item ROC AUC = 0.916
        \end{itemize}
        \end{footnotesize}
    \end{column}
    
    \begin{column}{0.5\textwidth}
    The 8 features of the simplified model are:
        \begin{footnotesize}
        \begin{enumerate}
            \item Arrival-to-scan time
            \item Stroke type (infarction/haemorrhage)
            \item Stroke severity (NIHSS)
            \item Precise or estimated stroke onset time
            \item Prior disability level (mRS)
            \item Stroke team
            \item Use of AF anticoagulants
            \item Onset-to-arrival time
        \end{enumerate}
        \end{footnotesize}
        
    \vspace{2mm}
    \tiny{There are only very weak correlations between the selected features with no R-squared being greater than 0.05.}
    \end{column}
    
\end{columns}
\end{frame}


%%%%%%%%%%%%%%%%%%%%%%%%%%%%%%%%%%%%%%%%%%%%%%%%%%%%%%%%%%%%%%%

\begin{frame}
\frametitle{Explaining model predictions with SHAP values}

SHAP values show the influence of features (even for \emph{`black box'} models).

\begin{center}
\includegraphics[width=0.85\textwidth]{./images/xgb_waterfall_low_probability.jpg}
\end{center}
\end{frame}

%%%%%%%%%%%%%%%%%%%%%%%%%%%%%%%%%%%%%%%%%%%%%%%%%%%%%%%%%%%%%%%

\begin{frame}
\frametitle{What drives use of thrombolysis across all hospitals?}

\footnotesize{Note: SHAP values here are \emph{log odds}. Each step-change in value of \textpm 1 changes the chances of receiving thrombolysis about 3-fold. (Plots are in order of feature importance.)}

\begin{center}
\includegraphics[width=0.80\textwidth]{./images/xgb_thrombolysis_shap_violin.jpg}
\end{center}
\end{frame}

%%%%%%%%%%%%%%%%%%%%%%%%%%%%%%%%%%%%%%%%%%%%%%%%%%%%%%%%%%%%%%%

\begin{frame}
\frametitle{Investigating how hospitals differ in thrombolysis decision-making (Patient 1: Base patient)}

Assuming there are no reasons not stated here to exclude a patient from use of thrombolysis, would you give this patient thrombolysis?
\vspace{3mm}

\begin{columns}
    \begin{column}{0.5\textwidth}
        \begin{itemize}
            \item Onset to arrival = 80 mins
            \item Arrival to scan = 20 mins
            \item Infarction = Yes
            \item NIHSS = 15
            \item Prior disability level = 0
            \item Precise onset time = Yes
            \item Use of AF anticoagulents = No
        \end{itemize}
    \end{column}
    
    \begin{column}{0.4\textwidth}
    Our model predicts 131 out of 132 (99\%) hospitals would give this patient thrombolysis.
    \end{column}

\end{columns}
\end{frame}

%%%%%%%%%%%%%%%%%%%%%%%%%%%%%%%%%%%%%%%%%%%%%%%%%%%%%%%%%%%%%%%

\begin{frame}
\frametitle{Investigating how hospitals differ in thrombolysis decision-making (Patient 2: Milder stroke)}

Assuming there are no reasons not stated here to exclude a patient from use of thrombolysis, would you give this patient thrombolysis?

\vspace{3mm}

\begin{columns}
    \begin{column}{0.5\textwidth}
        \begin{itemize}
            \item Onset to arrival = 80 mins
            \item Arrival to scan = 20 mins
            \item Infarction = Yes
            \item \emph{NIHSS = 4}
            \item Prior disability level = 0
            \item Precise onset time = Yes
            \item Use of AF anticoagulents = No
        \end{itemize}
    \end{column}
    
    \begin{column}{0.4\textwidth}
    Our model predicts 97 out of 132 (73\%) hospitals would give this patient thrombolysis.
    \end{column}

\end{columns}
\end{frame}

%%%%%%%%%%%%%%%%%%%%%%%%%%%%%%%%%%%%%%%%%%%%%%%%%%%%%%%%%%%%%%%

\begin{frame}
\frametitle{Investigating how hospitals differ in thrombolysis decision-making (Patient 3: Pre-stroke disability)}

Assuming there are no reasons not stated here to exclude a patient from use of thrombolysis, would you give this patient thrombolysis?

\vspace{3mm}

\begin{columns}
    \begin{column}{0.5\textwidth}
        \begin{itemize}
            \item Onset to arrival = 80 mins
            \item Arrival to scan = 20 mins
            \item Infarction = Yes
            \item NIHSS = 15
            \item \emph{Prior disability level = 3*}
            \item Precise onset time = Yes
            \item Use of AF anticoagulents = No
        \end{itemize}
    \vspace{3mm}    
    \footnotesize{*Moderate disability; requires some help, but able to walk without assistance.}
    \end{column}
    
    \begin{column}{0.4\textwidth}
    Our model predicts 114 out of 132 (86\%) hospitals would give this patient thrombolysis.
    \end{column}

\end{columns}
\end{frame}

%%%%%%%%%%%%%%%%%%%%%%%%%%%%%%%%%%%%%%%%%%%%%%%%%%%%%%%%%%%%%%%

\begin{frame}
\frametitle{Investigating how hospitals differ in thrombolysis decision-making (Patient 4: Estimated stroke onset time)}

Assuming there are no reasons not stated here to exclude a patient from use of thrombolysis, would you give this patient thrombolysis?

\vspace{3mm}

\begin{columns}
    \begin{column}{0.5\textwidth}
        \begin{itemize}
            \item Onset to arrival = 80 mins
            \item Arrival to scan = 20 mins
            \item Infarction = Yes
            \item NIHSS = 15
            \item Prior disability level = 0
            \item \emph{Precise onset time = No}
            \item Use of AF anticoagulents = No
        \end{itemize}
    \end{column}
    
    \begin{column}{0.4\textwidth}
    Our model predicts 84 out of 132 (64\%) hospitals would give this patient thrombolysis.
    \end{column}

\end{columns}
\end{frame}


%%%%%%%%%%%%%%%%%%%%%%%%%%%%%%%%%%%%%%%%%%%%%%%%%%%%%%%%%%%%%%%
\begin{frame}
\frametitle{Machine Learning key findings}
General observations about thrombolysis use: The chance of receiving thrombolysis is increased by:
\emph{
\begin{itemize}
    \item Shorter arrival-to-scan times
    \item Mid-level stroke severity
    \item Precise onset time
    \item Lower pre-stroke disability
\end{itemize}
}

\vspace{5mm}

Lower thrombolysing units are particularly less likely to give thrombolysis to patients with:
\emph{
\begin{itemize}
    \item Low or high stroke severity
    \item Higher pre-stroke disability
    \item Estimated onset time
\end{itemize}
}

\end{frame}

%%%%%%%%%%%%%%%%%%%%%%%%%%%%%%%%%%%%%%%%%%%%%%%%%%%%%%%%%%%%%%%
%%%%%%%%%%%%%%%%%%%%%%%%%%%%%%%%%%%%%%%%%%%%%%%%%%%%%%%%%%%%%%%
%%%%%%%%%%%%%%%%%%%%%%%%%%%%%%%%%%%%%%%%%%%%%%%%%%%%%%%%%%%%%%%
%%%%%%%%%%%%%%%%%%%%%%%%%%%%%%%%%%%%%%%%%%%%%%%%%%%%%%%%%%%%%%%

%%%%%%%%%%%%%%%%%%%%%%%%%%%%%%%%%%%%%%%%%%%%%%%%%%%%%%%%%%%%%%%
\section{Stroke outcome modelling based on times to treatment with thrombolysis and thrombectomy}



%%%%%%%%%%%%%%%%%%%%%%%%%%%%%%%%%%%%%%%%%%%%%%%%%%%%%%%%%%%%%%%

\begin{frame}{Model combines multiple data sources}


    
\end{frame}





%%%%%%%%%%%%%%%%%%%%%%%%%%%%%%%%%%%%%%%%%%%%%%%%%%%%%%%%%%%%%%%

% EXTRA SLIDE(S)

\begin{frame}{Reserve slides}
    
\end{frame}

%%%%%%%%%%%%%%%%%%%%%%%%%%%%%%%%%%%%%%%%%%%%%%%%%%%%%%%%%%%%%%%%%

\begin{frame}
\frametitle{When will low thrombolysing units not use thrombolysis when higher thrombolysing would?}

\footnotesize{Here, a high SHAP shows when a low-thrombolysing unit will reject use of thrombolysis when a higher thrombolysing hospital would use thrombolysis. (Plots are in order of feature importance.)}

\begin{center}
\includegraphics[width=0.75\textwidth]{./images/xgb_predicting_difference_shap_violin.jpg}
\end{center}
\end{frame}



\end{document}




